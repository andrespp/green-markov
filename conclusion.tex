\begin{frame}{Conclusões}
  \begin{block}{}
    \begin{itemize}
      \item \textit{Femtocells} permitem um \alert{aumento de cobertura}
      celular, além da \alert{diminuição} da carga em \textit{Macrocells}, a um
      \alert{baixo custo}, contudo:
      \begin{itemize}
        \item alocação ótima de clientes ainda é um problema em aberto
        \item ainda mais complexo quando consideradas questões de eficiência
        energética
      \end{itemize}
    \end{itemize}
  \end{block}
  \pause
  \begin{block}{}
    \begin{itemize}
      \item Através de alocação ótima, o artigo busca oferecer os
      \alert{melhores níveis de serviço}, ao mesmo tempo que visa
      \alert{maximizar a vida útil da bateria}
      \begin{itemize}
        \item considera características dos tipos de serviço (voz/dados)
      \end{itemize}
    \end{itemize}
  \end{block}
\end{frame}

\begin{frame}{Resultados}
  \begin{block}{Obervações}
    \begin{itemize}
      \item Conexões de \alert{voz} devem ser direcionadas para
      \alert{\textit{macrocells}}
      \begin{itemize}
        \item Maior capacidade de clientes
        \item Maior cobertura
        \item Menores indices de perdas
      \end{itemize}
      \item Conxões de \alert{dados} devem ser direcionadas para
      \alert{\textit{femtocells}}
      \begin{itemize}
        \item Maior largura de banda
        \item Apesar das perdas, atende requisitos de QoS para dados (TCP/IP)
      \end{itemize}
    \end{itemize}
  \end{block}
\end{frame}

\begin{frame}{Resultados}
  \begin{block}{Contribuições}
    \begin{enumerate}
      \item Proposta de um modelo de alocação ótima
      \item Consideração de aspectos de diferentes camadas (nível de sinal,
      eficiência energética)
      \item \textit{Green Markov Library}
    \end{enumerate}
  \end{block}
\end{frame}

%\begin{frame}
%  \begin{block}{}
%    \begin{itemize}
%      \item
%    \end{itemize}
%  \end{block}
%\end{frame}

%\begin{frame}
%  \begin{block}{}
%  \end{block}
%\end{frame}

%\begin{frame}
%  \begin{figure}[h]
%  	\begin{center}
%      \includegraphics [scale=0.3]{./Figures/Device-Estimates}
%     % \caption {Estimativa de dispositivos conectados à Internet.}
%  		%\label{fig:arq-imuno}
%  	\end{center}
%  \end{figure}
%\end{frame}

%\begin{frame}{Redes de Acesso}
%	\begin{figure}[!htb]
%		\centering
%		\subfloat[DSL]{
%			\includegraphics[height=3.5cm]{./Figures/DSLaccess}
%			\label{figdroopy}}
%		\quad %espaco separador
%		\subfloat[Cable]{
%			\includegraphics[height=3.5cm]{./Figures/CableAccess}
%			\label{figsnoop}}
%		%\caption{Subfiguras}
%		%\label{fig01}
%	\end{figure}
%\end{frame}

%\begin{frame}[fragile]
%\scriptsize
%\begin{verbatim}
%\end{verbatim}
%\end{frame}
